\section{\iflanguage{english}{Conclusion}{Fazit}}\label{sec:conclusion}

\subsection*{ToDos}

\todo[inline]{Select the correct language in the first rows of \texttt{main.tex}. The \% makes a line a comment.}
\todo[inline]{Generate a PDF (Visual Studio Code: click on the green arrow in the top right corner).}
\todo[inline]{Check the PDF (Visual Studio Code: click on the two rectangles with the magnifying glass in the top right corner).}
\todo[inline]{With CTRL pressed you can click on the PDF and jump to the corresponding place in your \texttt{.tex} files.}
\todo[inline]{Check the Errors. (Visual Studio Code: bottom left)
    There should be one warning because the first ToDo pushes down the second one: \texttt{LaTeX Marginpar on page 2 moved.}}
\todo[inline]{In the \texttt{main.tex} you can press ALT + SHIFT and the ARROW UP and DOWN keys to create a multiline cursor,
    comment out all the sections you are currently not working on and remove the multiline cursor with ESC.}
\todo[inline]{Keep the following examples and todos till the end and check them out sometimes.}

\todo[inline]{Adjust your License Notes to reflect the used programs.}
\todo[inline]{For Printing: Remove the \texttt{todonotes}-package in \texttt{setup.tex} to find all ToDos.}
\todo[inline]{For Printing: Choose the correct formating (search \texttt{print} in \texttt{setup.tex}).}

You can remove the glossary and the symbolslist to make it easier, especially in a Bachelor's Thesis.

\subsection{Tables and References}

\begin{table}
    \centering
    \caption{A nice table with a nice caption.}\label{tab:values_and_units}
    \begin{tabular}{lll}
        \toprule
        Value \& Unit                        & Value & Unit                           \\
        \midrule
        \SI{9.82}{\meter\per\second\squared} & 9.82  & \si{\meter\per\second\squared} \\
        \bottomrule
    \end{tabular}
\end{table}

\autoref{tab:values_and_units}

This is a cited book~\cite{latex-template}.

Store and display plots and other graphs as \texttt{.pdf} files.

Include \autoref{code:python-example} in the same way you include equations like \autoref{eq:example}:

\begin{align}\label{eq:example}
    y & = 3 x^2 + 5 x^2 + 4 x + 4 x + 7 + 1 \nonumber \\
      & = 8 x^2 +         8 x  +      8     \nonumber \\
    y & = 8 \cdot (x^2 +  x +         1 )
\end{align}

\subsection{Auto-complete in VSCode}

There are also some auto-complete features: type @ + letter to get the letters greek version.

Type @v + letter for the alternative form (like \(\varphi\) for \(\phi\)). % chktex 21

You can also use shortcuts for environments like
\begin{align}
    BAL
\end{align}

Type BAL and select the correct option from the autocomplete. Maybe you have to try various ones.
The B is the base command and AL is the abbreviation for align.

If you want an environment without an index, you can use BS as base command.

Same for changes to your font, like bold or italic:
FIT gives \textit{text}

\subsection{'', Quotations and Code}

Use '' or \texttt{file-names}, \verb|file_names| and \verb+code_snippets+.

\lstinputlisting[language=Python, label=code:python-example,
    caption={The Colors from the Corporate Design of the University of Freiburg.
            The keywords are bold and the comments blue. This can be changed in \texttt{setup.tex}.}
]{assets/corporate_design_colors.py}

\subsection{Clipboard Manager, Git and more}

Clipboard Manager help with copying multiple different things at once.
On Windows you can press WINDOWS + V to activate the history
and choose older things you copied.

Use \href{https://www.youtube.com/watch?v=hwP7WQkmECE&pp=ygUDZ2l0}{Git (click here)}.

\subsection{Glossary}

Click them, the page numbers in the glossaries are also clickable.

\gls{label-one}

\gls{label-one}

\gls{label-one}

\gls{label-two}

\gls{label-two}

\gls{label-two}

\gls{label-three}

\gls{label-three}

\gls{label-three}

\newpage

Big Letter at start: \Gls{label-one}

\glspl{label-one}

\gls{label-one}

Big Letter at start: \Gls{label-two}

\gls{label-two}

\glspl{label-two}

Big Letter at start: \Gls{label-three}

\gls{label-three}
